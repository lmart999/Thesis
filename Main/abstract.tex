
\prefacesection{Abstract}

Infectious diseases have a profound impact on humankind, influencing the course of wars and the fate of nations. The case for next-generation sequencing (NGS) in modern clinical microbiology is becoming increasingly clear. A flood of recent studies have shown how this powerful technology can address pressing modern problems in infectious diseases, including resistance, rare infections, and outbreaks.

In this thesis, we show that molecular counting of pathogen-derived cell-free DNA is a new diagnostic paradigm for infectious disease. We built a pipeline for counting pathogen-derived cell-free molecules in human plasma and a web application presenting the resulting data. We applied these tools to thousands of clinical samples collected from hundreds of patients at Stanford hospital. We further processed thousands of clinical test records in order to show that this method can be broadly applied for non-invasive monitoring of viral, bacterial, and fungal infections in deep tissues. Finally, we show that unbiased pathogen monitoring using this technique can track infections that escape hypothesis-centric clinical testing.

After demonstrating this new diagnostic application of NGS, we show how NGS technology can be used to understand infectious disease mechanism. We developed a pipeline for counting of sequencing reads derived from RNA-protein interactions \emph{in vivo}. We show that this method (CLIP-seq) can be applied to viruses that have infected human cells and use it to reveal novel interactions between the HCV genome and human protein PCBP2. In a follow-up study, we applied the method to HERV-K, an endogenous retrovirus. We showed that human embryo development occurs in the presence of retroviral products, which protects the embroyo from exogenous infection while exerting regulatory function through interaction with human mRNAs. 

We highlight three separate ways to validate results from mechanistic CLIP-seq experiments, including comparative analysis, replicate matching, and functional studies. We also developed a highly multiplexed RNA-protein interaction assay that is compatible with the scale of CLIP-seq experiments and far exceeds the throughput of common biochemical assays. We applied this technology to a model RNA-protein interaction (histone stem-loop and SLBP), recapitulating two decades of biochemistry in a single experiment while also revealing novel features of the interaction.  

In summary, we built two computational tools that apply molecular counting to infectious disease diagnostics and mechanism. For validation of these results, we developed novel microfluidic tool for high-throughput biophysical measurments.